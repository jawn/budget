\documentclass[12pt,letterpaper]{article}
\usepackage{listings,lstautogobble}
\usepackage{framed}
\usepackage{xcolor}
\definecolor{navyblue}{rgb}{0.0, 0.0, 0.502}
\definecolor{darkgreen}{rgb}{0.04, 0.196, 0.125}
\definecolor{lightlightgray}{rgb}{0.9, 0.9, 0.9}
\definecolor{darkgray}{rgb}{0.2, 0.2, 0.2}
\definecolor{royalblue}{cmyk}{1, 0.50, 0, 0}
\definecolor{forestgreen}{rgb}{0.133, 0.545, 0.133}
\renewcommand{\familydefault}{\sfdefault}
\lstdefinelanguage{budget}{
autogobble=true,
basicstyle=\color{forestgreen}\ttfamily\small,
keywordstyle=[0]\color{black}\bf,
keywordstyle=[1]\color{navyblue},
keywordstyle=[2]\color{royalblue},
keywords=[0]{budget},
keywords=[1]{summary, detail, import, help, sum, det, imp, h, su},
keywords=[2]{transactions, categories, category, period, month, sortby, trans, cat, per, mo, c, s, M},
}
\lstnewenvironment{budget}[1][]
{
    \lstset{language={},
    autogobble=true,
    linewidth=4.5in,
    backgroundcolor=\color{lightlightgray},
    basicstyle=\color{forestgreen}\ttfamily\small,
    keywordstyle=[0]\color{black}\bf,
    keywordstyle=[1]\color{navyblue},
    keywordstyle=[2]\color{royalblue},
    keywords=[0]{budget},
    keywords=[1]{summary, detail, import, help, sum, det, imp, h, su},
    keywords=[2]{transactions, categories, category, period, month, sortby, trans, cat, per, mo, c, s, M},
#1}}{\vspace{0.2in}}    


\begin{document}
\Large
\begin{framed}
    \begin{minipage}[t][3in][t]{5in}
        You can abbreviate every command and option! \\ \\
        the command:
        \begin{budget}
            budget detail category Groceries sortby M
        \end{budget}
        is equivalent to: 
        \begin{budget}
            budget det cat Groceries sor M
        \end{budget}
        is equivalent to: 
        \begin{budget}
            budget d c Groceries s M
        \end{budget}
    \end{minipage}
\end{framed}
\begin{framed}
    \begin{minipage}[t][3in][t]{5in}
        You can have a summary of your favorite categories 
        \large
        \begin{enumerate}
            \item write your categories in a CSV file
                \begin{itemize}
                    \item one category per line
                    \item no quotes
                \end{itemize}
                \begin{budget}
                    Groceries
                    Business Expenses
                    Credit Card Payments
                \end{budget}
            \item use the \lstinline[language=budget,basicstyle=\large]!categories! options:
                \begin{budget}
                    budget summary categories MyFavorite.csv 
                \end{budget}
        \end{enumerate}
    \end{minipage}
\end{framed}
\begin{framed}
    \begin{minipage}[t][3in][t]{5in}
        \Large
        You can select data for a given period of time\\

        use the \lstinline[language=budget,basicstyle=\Large]!period! option followed by two dates
                \begin{itemize}
                    \item dates should be in the format \texttt{MM/DD/YYYY}
                    \item in which order doesn't matter
                \end{itemize}
                \begin{budget}
                    budget detail period 01/01/2020 01/31/2020

                    budget summary p 10/17/2020 01/23/2020 
                \end{budget}
    \end{minipage}
\end{framed}
\begin{framed}
    \begin{minipage}[t][3in][t]{5in}
        \Large
        You can select data for a given month\\

        use the \lstinline[language=budget,basicstyle=\Large]!month! then the year and month\\
        \begin{budget}
                    budget summary month 2020 03

                    budget detail mo 2020 12
                \end{budget}
    \end{minipage}
\end{framed}
\begin{framed}
    \begin{minipage}[t][3in][t]{5in}
        \Large
        You can sort transactions with several criteria\\

        use the \lstinline[language=budget,basicstyle=\Large]!sortby! option followed by letters
        \begin{itemize}
            \item \textbf{D} for date, \textbf{M} for amount, \textbf{C} for category
                \item \textbf{N} for name, \textbf{A} for account, \textbf{O} for note
            \item uppercase letter means ascending order
            \item lowercase letter means descending order
        \end{itemize}
        \begin{budget}
            budget detail sortby CDm
                \end{budget}
    \end{minipage}
\end{framed}
\begin{framed}
    \begin{minipage}[t][3in][t]{5in}
        \Large
        You can sort the summary by category or amount\\

        use the \lstinline[language=budget,basicstyle=\Large]!sortby! option followed by letters
        \begin{itemize}
            \item \textbf{C} for category
                \item \textbf{M} for amount
            \item uppercase letter means ascending order
            \item lowercase letter means descending order
        \end{itemize}
        \begin{budget}
            budget summary sortby m
                \end{budget}
    \end{minipage}
\end{framed}
\begin{framed}
    \begin{minipage}[t][3in][t]{5in}
        \Large
        The rules for importing transactions:\\
        \normalsize
        \begin{itemize}
            \item files containing transactions that are already imported are rejected
            \item a transaction is already imported if there is already a transaction with the same date, name, and amount in the main transaction file 
            \item transactions with a status different from "posted" are not imported 
            \item transactions with a status "posted" are imported and this info is replaced with the account name
            \item transactions where the account name is already set are imported, but their account name is unchanged.
        \end{itemize}

    \end{minipage}
\end{framed}
\end{document}
